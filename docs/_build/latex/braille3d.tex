%% Generated by Sphinx.
\def\sphinxdocclass{report}
\documentclass[letterpaper,10pt,english]{sphinxmanual}
\ifdefined\pdfpxdimen
   \let\sphinxpxdimen\pdfpxdimen\else\newdimen\sphinxpxdimen
\fi \sphinxpxdimen=.75bp\relax

\PassOptionsToPackage{warn}{textcomp}
\usepackage[utf8]{inputenc}
\ifdefined\DeclareUnicodeCharacter
% support both utf8 and utf8x syntaxes
  \ifdefined\DeclareUnicodeCharacterAsOptional
    \def\sphinxDUC#1{\DeclareUnicodeCharacter{"#1}}
  \else
    \let\sphinxDUC\DeclareUnicodeCharacter
  \fi
  \sphinxDUC{00A0}{\nobreakspace}
  \sphinxDUC{2500}{\sphinxunichar{2500}}
  \sphinxDUC{2502}{\sphinxunichar{2502}}
  \sphinxDUC{2514}{\sphinxunichar{2514}}
  \sphinxDUC{251C}{\sphinxunichar{251C}}
  \sphinxDUC{2572}{\textbackslash}
\fi
\usepackage{cmap}
\usepackage[T1]{fontenc}
\usepackage{amsmath,amssymb,amstext}
\usepackage{babel}



\usepackage{times}
\expandafter\ifx\csname T@LGR\endcsname\relax
\else
% LGR was declared as font encoding
  \substitutefont{LGR}{\rmdefault}{cmr}
  \substitutefont{LGR}{\sfdefault}{cmss}
  \substitutefont{LGR}{\ttdefault}{cmtt}
\fi
\expandafter\ifx\csname T@X2\endcsname\relax
  \expandafter\ifx\csname T@T2A\endcsname\relax
  \else
  % T2A was declared as font encoding
    \substitutefont{T2A}{\rmdefault}{cmr}
    \substitutefont{T2A}{\sfdefault}{cmss}
    \substitutefont{T2A}{\ttdefault}{cmtt}
  \fi
\else
% X2 was declared as font encoding
  \substitutefont{X2}{\rmdefault}{cmr}
  \substitutefont{X2}{\sfdefault}{cmss}
  \substitutefont{X2}{\ttdefault}{cmtt}
\fi


\usepackage[Bjarne]{fncychap}
\usepackage{sphinx}

\fvset{fontsize=\small}
\usepackage{geometry}


% Include hyperref last.
\usepackage{hyperref}
% Fix anchor placement for figures with captions.
\usepackage{hypcap}% it must be loaded after hyperref.
% Set up styles of URL: it should be placed after hyperref.
\urlstyle{same}
\addto\captionsenglish{\renewcommand{\contentsname}{Contents:}}

\usepackage{sphinxmessages}
\setcounter{tocdepth}{1}



\title{Braille3d}
\date{May 19, 2020}
\release{}
\author{Cole Blackman}
\newcommand{\sphinxlogo}{\vbox{}}
\renewcommand{\releasename}{}
\makeindex
\begin{document}

\pagestyle{empty}
\sphinxmaketitle
\pagestyle{plain}
\sphinxtableofcontents
\pagestyle{normal}
\phantomsection\label{\detokenize{index::doc}}



\chapter{Introduction}
\label{\detokenize{Introduction:introduction}}\label{\detokenize{Introduction::doc}}
TODO


\section{What this project is}
\label{\detokenize{Introduction:what-this-project-is}}
Braille3d is a set of documentation and resources for constructing a braille printer capable of printing a4, blind\sphinxhyphen{}legible documents.
A printer constructed using this project should be:
\begin{itemize}
\item {} 
Cost\sphinxhyphen{}effective

\item {} 
Extensible

\item {} 
Easy to maintain

\item {} 
Simple and reliable

\end{itemize}

This is a project to redesign and complete the original French BrailleRap\sphinxhyphen{}SP documentation. All credit for the design of the printer goes to: Philippe Pacotte and Stéphane Godin.

Unfortunately, progress on the initial project has stalled, so this is an effort to pick up where the original team left off.


\section{Who it is for}
\label{\detokenize{Introduction:who-it-is-for}}
This project should be doable by a team of students or hobbyists with access to the following:
\begin{itemize}
\item {} 
A laser cutter or some other way to cut the plywood according to spec

\item {} 
A 3d printer

\item {} 
Basic shop tools such as an electric drill and screwdrivers

\item {} 
The materials listed in the Materials section

\end{itemize}

When all materials have been sourced, the project should require several weeks to construct.


\section{Where to get help}
\label{\detokenize{Introduction:where-to-get-help}}\begin{enumerate}
\sphinxsetlistlabels{\arabic}{enumi}{enumii}{}{.}%
\item {} 
Consult this documentation

\item {} 
Contact us {[}TODO: add contact info{]}

\end{enumerate}


\section{A note on BrailleRap\sphinxhyphen{}SP vs Braille3d}
\label{\detokenize{Introduction:a-note-on-braillerap-sp-vs-braille3d}}
BrailleRap\sphinxhyphen{}SP is the original project this documentation is based on, though the designs have been modified somewhat.
Braille3d is an effort to complete the documentation of BrailleRap.

{[}TODO: finish these sections{]}
how this text is organized

get this text as a pdf

project history


\section{An overview of the design}
\label{\detokenize{Introduction:an-overview-of-the-design}}

\chapter{Materials List}
\label{\detokenize{Materials:materials-list}}\label{\detokenize{Materials::doc}}
{[}TODO{]} this should have descriptions of what each part is used for, and a column detailing where to acquire it (print/buy/etc) and if lathing them is a possibility

A note on acquiring materials:

Some of the required parts for this project would most easily be 3d printed. These parts are listed under the 3d Printing section.


\section{Frame}
\label{\detokenize{Materials:frame}}
The frame is plywood, commonly referred to as laser plywood (any plywood with roughly the correct thickness will work). We recommend 5mm width, though you may have success with 6mm sheets as well.

The sheet(s) should be large enough to cut the following plans from: {[}TODO: ADD LINK{]}

Sourcing the material: This should be very easy to acquire. Look for 2 sheets of 600mm x 400mm, such as here:
\sphinxurl{https://www.kitronik.co.uk/3202-6mm-laser-plywood-600mm-x-400mm-sheet.html}


\section{3d Printing}
\label{\detokenize{Materials:d-printing}}
Print settings: ABS, 50\% infill, 3 outside perimeters.

{[}TODO: add chart with links to each model{]}


\begin{savenotes}\sphinxattablestart
\centering
\begin{tabulary}{\linewidth}[t]{|T|T|T|T|}
\hline
\sphinxstyletheadfamily 
Printed Part
&\sphinxstyletheadfamily 
Image
&\sphinxstyletheadfamily 
Quantity
&\sphinxstyletheadfamily 
Usage
\\
\hline
KP08 Support
&
\noindent\sphinxincludegraphics{{KP08_support}.png}
&
2
&
This part is used to support the KP08 bearing. {[}TODO: add link to relevant build section{]}
\\
\hline
jfkdlsajflkdjsalkfjdsalkjflkdsajflkdjsa
&
jfkdlsajflkdsajflkjdsalkfjdlksajflkdsaj
&&\\
\hline
Scroll wheel
&&&\\
\hline
\end{tabulary}
\par
\sphinxattableend\end{savenotes}

KP08 Support


\section{Hardware components}
\label{\detokenize{Materials:hardware-components}}
{[}TODO: List hardware components{]}


\section{Electronics}
\label{\detokenize{Materials:electronics}}

\section{Board}
\label{\detokenize{Materials:board}}

\chapter{Building Guide}
\label{\detokenize{Build:building-guide}}\label{\detokenize{Build::doc}}
This building guide is divided into 3 main sections:

\DUrole{xref,std,std-ref}{External Assembly}

\DUrole{xref,std,std-ref}{Internal Assembly}

\DUrole{xref,std,std-ref}{Electronics Assembly}


\section{External Assembly}
\label{\detokenize{Build:external-assembly}}

\subsection{Frame}
\label{\detokenize{Build:frame}}
Material needed:

All the plywood frame pieces:
\begin{itemize}
\item {} 
Face

\item {} 
Back

\item {} 
Bottom

\item {} 
Left

\item {} 
Right

\end{itemize}

Make sure the holes in the plywood pieces are arranged like the following picture. The rounded pegs will become feet, so make sure they are facing downward. These pegs are identified with red dots.

\phantomsection\label{\detokenize{Build:center}}
\noindent{\hspace*{\fill}\sphinxincludegraphics{{Frame_layout}.jpg}\hspace*{\fill}}

You can now glue the pieces together with wood glue, but you may want to wait until the internal components are assembled. This can make it easier to work with assemblies like the guide rods. So, for now just snap the 5 pieces together and tape the corners as seen in \sphinxcode{\sphinxupquote{taped}}:

\phantomsection\label{\detokenize{Build:taped}}
\noindent{\hspace*{\fill}\sphinxincludegraphics{{Frame_taped}.jpg}\hspace*{\fill}}

{[}TODO: collage of trap blockers??{]}


\section{Internal Assembly}
\label{\detokenize{Build:internal-assembly}}

\subsection{Y\sphinxhyphen{}direction Motor Assembly}
\label{\detokenize{Build:y-direction-motor-assembly}}
Materials needed:
\begin{itemize}
\item {} \begin{description}
\item[{Y motor support:}] \leavevmode\phantomsection\label{\detokenize{Build:ymotor}}
\noindent\sphinxincludegraphics[width=200\sphinxpxdimen]{{YMotorsupport}.png}

\end{description}

\item {} 
2 M3 washers

\item {} 
2 M3\sphinxhyphen{}12 screws

\item {} 
2 Nylstop M3 Hex nuts

\item {} 
4 M3\sphinxhyphen{}8 screws

\item {} 
1 GT2 pulley, 20 teeth, 5mm

\item {} 
1 Nema 17 motor

\end{itemize}


\bigskip\hrule\bigskip

\begin{enumerate}
\sphinxsetlistlabels{\arabic}{enumi}{enumii}{}{.}%
\item {} 
Add the Y Motor support to the rotating end of the Nema motor, with the motor power connector facing upwards as shown.

\end{enumerate}

\noindent\sphinxincludegraphics{{YSupport_install}.jpg}
\begin{enumerate}
\sphinxsetlistlabels{\arabic}{enumi}{enumii}{}{.}%
\setcounter{enumi}{1}
\item {} 
Place the hex nuts into the hex nut grooves in the 3d\sphinxhyphen{}printed Y motor support.

\end{enumerate}
\begin{enumerate}
\sphinxsetlistlabels{\arabic}{enumi}{enumii}{}{.}%
\setcounter{enumi}{2}
\item {} 
Attach the Y\sphinxhyphen{}motor assembly to the frame. Take care that it is facing in the correct direction relative to the holes in the bottom of the frame, as shown.

\end{enumerate}

{[}TODO: Complete section{]}


\section{Electronics Assembly}
\label{\detokenize{Build:electronics-assembly}}
{[}TODO: Complete section{]}

This is a project to redesign and complete the original French BrailleRap\sphinxhyphen{}SP documentation. All credit for the design of the printer goes to: Philippe Pacotte and Stéphane Godin.

Unfortunately, progress on the initial project has stalled, so this is an effort to pick up where the original team left off.

The original documentation was released under the CERN Open Hardware License, as is this repository. Thus, the all documentation is free to study, modify, and share.


\chapter{About the project}
\label{\detokenize{index:about-the-project}}
The goal of this project is to produce complete and easily reproducible documentation for construction of a braille printer/embosser for the production of blind\sphinxhyphen{}legible texts.

It is based on, and shares much of its design and layout with, BrailleRap\sphinxhyphen{}SP. At times, however, the design, layout, and models diverge from the original BrailleRap\sphinxhyphen{}SP, and as such it is a distinct project.

Further goals of the project include:
\sphinxhyphen{} Complete English documentation
\sphinxhyphen{} Modifying of designs to accommodate more easily\sphinxhyphen{}acquired parts
\sphinxhyphen{} Beginner\sphinxhyphen{}oriented and student\sphinxhyphen{}accessible documentation
\sphinxhyphen{} Simplification of design choices


\chapter{Indices and tables}
\label{\detokenize{index:indices-and-tables}}\begin{itemize}
\item {} 
\DUrole{xref,std,std-ref}{genindex}

\item {} 
\DUrole{xref,std,std-ref}{modindex}

\item {} 
\DUrole{xref,std,std-ref}{search}

\end{itemize}



\renewcommand{\indexname}{Index}
\printindex
\end{document}